%% Mousse au Chocolat Example

\begin{otherlanguage}{ngerman}

\setHeadlines
{% translation
    inghead = Zutaten,
    prephead = Zubereitung,
    hinthead = Tipp,
    continuationhead = Fortsetzung,
    continuationfoot = Fortsetzung auf n\"achster Seite,
    portionvalue = Personen,
}

\begin{recipe}
[ % Optionale Eingaben
    preparationtime = {\unit[1]{h}},
    portion = \portion{5},
    source = R. Gaus
]
{Mousse au Chocolat}
    
    \graph
    {% Bilder
        small=pic/glass,    % kleines Bild
        big=pic/ingredients % gro�es (l�ngeres) Bild
    }
    
    \ingredients
    )\\
        3 & Eier\\
        \unit[200]{ml} & Sahne\\
        \unit[40]{g} & Zucker\\
        \unit[50]{g} & Butter
    }
    
    \preparation
    { % Zubereitung
        \step Eier trennen, Eiwei� und Sahne separat steif schlagen. Butter und Schokolade vorsichtig im Wasserbad schmelzen.
        \step Eigelb in einer gro�en Sch�ssel mit \unit[2]{EL} hei�em Wasser cremig schlagen, den Zucker einr�hren bis die Masse hell und cremig ist.
        \step Die geschmolzene Schokolade unterheben, anschlie�end sofort Eischnee und Sahne unterheben (nicht mit dem Elektro-Mixer!)
        \step Mindestens 2 Stunden im K�hlschrank kalt stellen. Aber nicht zu kalt servieren.
    }
    
    \hint
    {% Tipp
        Der Schokoladenanteil kann auch gesenkt werden.
    }

\end{recipe}

\end{otherlanguage}