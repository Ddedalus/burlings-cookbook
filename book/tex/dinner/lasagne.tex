% Complete recipe example
\begin{recipe}
[% 
    preparationtime = {\unit[1]{h}},
    bakingtime={\unit[1]{h}},
    portion = {\portion{4-5}}
]
{Lasagne with pumpkin and millet}

 \introduction{%
	It may seem like a lot of faff but trust me, it's worth it! Pumpkin-millet filling is insanely creamy and capars together with dried tomatoes add extra zest. From now on, pumpkin is your reason to await Halloween! 
}
    
    \ingredients{%
        \unit[2]{c} & Pumpkin purée \\
        \unit[3/4]{c} & Millet \\
        \unit[200]{ml} & Cream \\
        Handful & Capers \\
        Handful & Dried tomatoes \\
        & Sage \\
        Box	& Lasagne sheets \\
        \textbf{Sauce} & \\
        2 cans & Tomatoes \\
        \unit[400]{ml} & Passata \\
        1 & Onion \\
        & Smoked paprika \\
        & Balsamic vinegar \\
        & \\
        & Cheese for topping - mozzarella or grated cheddar
    }
    
    \preparation{%
        \step Rinse millet with boiling water. Add 1.5 c of water, salt and cook at small heat till millet absorbed all liquid. If still not tender, add more water and cook till tender. When cooking millet, leave it alone, don't stir it, don't uncover it etc. 
        \step Fry onion, add all sauce ingredients and simmer for as long as you can.
        \step Mix with blender cooked millet, pumpkin purée and remaining filling ingredients apart for capers and dried tomatoes. Add capers and chopped dried tomatoes.  
        \step In high oven dish layer alternately sauce, lasagne sheets, pumpkin filling and lasagne sheet. End with sauce, add cheese on the top.
        \step Bake for about an 1h (check if pasta is tender).
    }
    
    \suggestion[]
    {%
       You can also prepare béchamel sauce and either alternate it with tomato sauce or replace. 
    }
    
    
    \hint{%
        For extra umami flavour, add yeast flakes to the pumpkin filling or cook millet in broth. 
    }
    
\end{recipe}