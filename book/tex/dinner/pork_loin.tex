\begin{recipe}
    [% 
        preparationtime = {\unit[1]{h}},
        portion = {\portion{3-5}},
        source = {Magda's first cookbook}
    ]
    {Pork loin with baby carrots}

    \setRecipeLengths{
        ingredientswidth=5.5cm
    }

    \ingredients[17]{%
        \unit[200]{g} & bulgur or millet \\
        & \textbf{Carrots} \\
        \unit[500]{g} & baby carrots \\
        \unit[50]{g} & butter \\
        \unit[2]{tbs.} & honey \\
        a bit & chilli powder \\
        a lot & cinnamon \\
        & powdered ginger \\
        & \textbf{Pork} \\
        \unit[500]{g} & pork loin \\
        3-5 claws & garlic \\
        & flour \\
        & paprika \\
        & smoked paprika \\
        & soy sauce \\
        & herb pepper \\
        & salt\&pepper
    }

    \preparation{%
        \step Melt the butter and fry the carrots for 5 minutes, stirring.
        They should cover the whole pan.
        Pour boiling water to cover the carrots, put a lid on.

        \step when the carrots get soft-ish, add honey and spices, stir.
        Evaporate the water and let the honey caramelise.

        \step Boil the millet in salted water or vegetable stock.
        Stir in some butter if it is ready too early.

        \step Cut the pork loin into \unit[10-15]{mm} wide slices.
        You may smash it a bit.
        % TODO: vocab smash

        \step Mix all remaining spices (except for garlic) with a small amount of flour.
        The soy sauce should make the seasoning sticky but not watery.
        Cover the meat with all the seasoning.

        \step preheat a pan with a fair amount of oil (strong heat).
        Fry both sides of the pork.
        Add whole garlic claws to the pan to reduce oil sparkling.
        Replace them if burned.
        % TODO: vocab strong heat

        \step Pour in some boiling water and let it simmer until the meat is soft.
        Evaporate the water and serve in its own juice.
    }

\end{recipe}