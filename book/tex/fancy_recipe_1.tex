%% Mousse au Chocolat Example

% \begin{otherlanguage}{ngerman}

\setHeadlines
{% translation
    inghead = Zutaten,
    prephead = Zubereitung,
    hinthead = Tipp,
    continuationhead = Fortsetzung,
    continuationfoot = Fortsetzung auf nachster Seite,
    portionvalue = Personen,
}

\begin{recipe}
[ % Optionale Eingaben
    preparationtime = {\unit[1]{h}},
    portion = \portion{5},
    source = R. Gaus
]
{Mousse au Chocolat}

    \ingredients
    )\\
        3 & Eier\\
        \unit[200]{ml} & Sahne\\
        \unit[40]{g} & Zucker\\
        \unit[50]{g} & Butter
    }
    
    \preparation
    { % Zubereitung
        \step Eier trennen, Eiwei und Sahne separat steif schlagen. Butter und Schokolade vorsichtig im Wasserbad schmelzen.
        \step Eigelb in einer groen Schssel mit \unit[2]{EL} heiem Wasser cremig schlagen, den Zucker einrhren bis die Masse hell und cremig ist.
        \step Die geschmolzene Schokolade unterheben, anschlieend sofort Eischnee und Sahne unterheben (nicht mit dem Elektro-Mixer!)
        \step Mindestens 2 Stunden im Khlschrank kalt stellen. Aber nicht zu kalt servieren.
    }
    
    \hint
    {% Tipp
        Der Schokoladenanteil kann auch gesenkt werden.
    }

\end{recipe}

% \end{otherlanguage}