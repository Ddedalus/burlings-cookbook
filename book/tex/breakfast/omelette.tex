\begin{recipe}
    [%
        preparationtime = {\unit[5]{min}},
        portion = {\portion{1}},
        bakingtime = {\unit[10]{min}}
    ]
    {Omelette}
    \introduction{%
        This recipe is for one serving.
        However you can double the ingredients and make one, big omelette for more people (can be tricky to fry, though).
        What I usually do: instead of 'omelette batch' I prepare simultaneously two omelettes (or more) - two bowls with beaten eggs, two frying pans, etc...
    }

    \ingredients{%
        2-3 & Eggs \\
        0.25 c. & Milk \\
        Handf. & Spinach \\
        30 g & Feta cheese \\
        5 & Cherry tomatoes \\
        5 & Olives \\
        Handf. & Grated cheddar \\
        & Smoked paprica \\
        & Salt\&pepper \\
        & Butter to fry
    }

    \preparation{%
        \step Beat eggs, add milk, salt\&pepper and mix again.

        \step On preheated frying pan melt butter and pour egg mixture.
        Let is settle for 30s, mix gently not congealed part (don't disturb the bottom layer!).
        Immediately add all toppings and stir them in.
        Sprinkle with smoked paprika

        \step After 1 min, turn down the heat.
        Cover with lid.
        After 2-3 min fold the omelette in half, add a little bit of boiling water on the pan and cover with lid (create steam so inner part of the omelette can cook quicker).

        \step After 2-3 min, sprinkle with smoked paprika, turn (if possible) for extra 30s.
        The omelette is done!
    }

    \hint{%
        Instead of folding and playing with steam, you can simply turn the omelette and fold it after cooking.
    }

\end{recipe}