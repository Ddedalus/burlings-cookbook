\begin{recipe}
    [% 
        preparationtime = {\unit[0.5]{h}},
        bakingtime={\unit[0.5]{h}},
        portion = {\portion{4}}
    ]
    {Risotto}

    \introduction{%
        People still haven't figured out why exactly stirring makes risotto so delicious. There are a few theories, covering even spread of heat and increased 'leakage' of starch from rice grains. Regardless of the reason, bear in mind that stirring, keeping the broth hot at all time and adding it in small portions are essential!
    }

    \ingredients{%
        350 g  & Risotto rice \\
        Bunch  & Asparagus \\
        3 & Red/yellow peppers \\
        1 & Red onion \\
        500 ml & Broth/stock \\
        200 ml& Apple cider \\
        & Cheese (Parmesan or mature cheddar) \\
        250 g & Pancetta (Smoked)
    }

    \preparation{%
        \step Heat up broth and keep hot.
        \step Fry finely diced onion. Add rice and fry for 2-3 minutes.
        \step Add alternately broth and cider in small portions (ladle or two) at a time. Keep stirring rice. Add more liquid when previous portion almost absorbed.
        \step In meantime roast peppers and fry asparagus (cut in 3 pieces) or fry both. Fry pancetta.
        \step Stir vegetable and pancetta in. Add cheese.

    }

    \suggestion[]
    {%
        The larger the pan, the less stirring required as rice is heated more evenly.
    }

\end{recipe}