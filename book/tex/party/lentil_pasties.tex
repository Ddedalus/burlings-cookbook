
\begin{recipe}
    [% 
        preparationtime = {\unit[40]{minutes} +  \unit[2]{h} waiting},
        bakingtime = {\unit[30]{min}},
        portion = {40-50 pasties}
    ]
    {Lentil pasties}

    \introduction{%
        These make a marvelous lunch, especially when you go for a whole day trip.
        As meat-free you can store them in a warm backpack and then eat without fear...

        The pastry is universal and versatile - you will always succeed.
        The tricky bit is the stuffing - you need to make sure it stands out.
        Nutmeg is your friend here, go for a lot.
        Also, make sure it is rich with umami taste: mushrooms, dried tomatoes, soy sauce are all good sources of it.

    }

    \ingredients{%
        \textbf{Pastry} & \\
        1 glass & milk \\
        3 glasses & flour \\
        & dried yiest \\
        2 tsp & (cane) sugar \\
        1/2 tsp & salt \\
        1/3 glass & oil \\
        \textbf{Stuffing} & \\
        1 glass & green (brown) lentil \\
        & dried mushrooms \\
        1 & large onion \\
        2 & bay leafs \\
        2 grains & allspice \\
        2 & cloves \\
        & juniper \\
        % TODO: http://www.jadlonomia.com/przepisy/paszteciki-z-soczewica-2/
        5 Tsp & oil \\
        2 Tsp & soy sauce \\
        2 Tsp & milk \\
        & nutmeg \\
        & salt \& pepper
    }

    \preparation{%
        \step Warm up the milk a little bit.
        Add it to the dry ingredients and pug for 3-4 minutes.
        Then add oil and pug for another 2-3 minutes unitl you form a flexible ball.
        Lat it rest in a warm place for 1-1.5 hours.
        \step Boil the mushrooms, lentils and oil in salted water for \unit{18-20}[minutes].
        With water aim for about double or tripple the amount of lentils (mass).
        \step Chop the onion, fry it for 5 minutes, adding all the dry spices.
        When it becomes soft, take out the bay leaf and allspice.
        \step Rinse the lentils if needed, mix with the onion, add the soy sauce and blend all that into a smooth paste.
        Cool down by spreading on a large plate.
        \step When the pastry has grown, pug it for 2 minutes and roll out 3 or 4 long strips \unit[6-8]{cm} wide and \unit[0.5]{cm} thick.
        \step Put the stuffing along the middle of each strip and join the sides of the pastry to form a roll.
        Now roll the tube so that the joint is at the bottom.
        Slice the rolls into \unit[2-3]{cm} wide pasties.
        \step Put the pasties on some baking paper and leave them to grow for 30 minutes.
        Make sure to preheat the oven properly to \unit[180]{\textcelcius}.
        Spread some milk on on pasties (to make them crunchy but not burned) and bake for 25-30 minutes.
    }

\end{recipe}