\begin{recipe}
    [% 
        preparationtime = {\unit[30]{min}},
        bakingtime = {\unit[60]{min}}
    ]
    {Tofu pumpkin 'cheesecake'}
    \introduction{%
        This recipe it's quite time-consuming.
        You have to cook millet, prepare pumpkin purée and gather all the ingredients.
        The cake is, however, spectacular, so let's celebrate!

        Coconut milk must be from the can as this one is rich in fat that adds mellowness to the cake.

        Millet makes the cake more dense and less spongy.

        Lemon juice is essential to turn tofu into more 'cheesy' base.

        Long blending is essential for creamy texture!
    }

    \ingredients{%
        \unit[150]{g} & Digestives \\
        3 tbs. & Peanut butter \\
        Pinch & Salt \\
        & \\
        \unit[360]{g} & Natural tofu \\
        \nicefrac{1}{2} c. & Pumpkin purée \\
        0.75 c. & Cooked millet \\
        0.75 c. & Caster sugar \\
        2 tbs. & Starch \\
        1 ts. & Cinnamon \\
        1 ts. & Cardamom \\
        1 ts. & Ginger \\
        \nicefrac{1}{4} ts. & Nutmeg \\
        &  \\
        1 c. & Coconut milk (from the can) \\
        \nicefrac{1}{4} c. & Orange juice \\
        \nicefrac{1}{4} c. & Lemon juice \\
        & \\
        75 g & Dark chocolate \\
        \nicefrac{1}{2} c. & Coconut milk (can)*
    }

    \preparation{%
        \step \textbf{Base}: blend negligently.
        Press into lined with parchment cake tin.
        Refrigerate.
        \step \textbf{Cake}: Blend thoroughly everything apart from: milk, lemon and orange juice.
        When dough is smooth, gradually add liquids.
        \step Pour onto the base.
        Bake at\underline{ \unit[180]{\textcelcius}} for 15 min.
        \step Then, turn the oven down to  \underline{\unit[120]{\textcelcius}} and bake for about 45 min.
        Switch the oven off but \underline{leave the cake in} for another 15 min.
        Cool down for 2h.
        \step \textbf{Coating}: Heat all ingredients in a small pot, till chocolate melts.
        Leave for 20 min to cool down, pour over \underline{cold} cake.
        \vspace{35mm}
    }
    \suggestion
    {%
        *For chocolate coating, you can use single cream instead but reduce the amount to about 1/3 cup.
        For sweeter coating, add caster sugar.
    }

    \hint{%
        When you turn temperature down to 120, you can add oven dish full of close to boiling temperature water at the bottom of the oven.
        Steam will stop the cake from cracking.
    }

\end{recipe}