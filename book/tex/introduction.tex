% !TEX root = ..\intro_example.tex

Dear Bethan and Hugh!

\section*{Welcome!}
When we were brainstorming on what to give you on this very special day we thought that it should consist of part of us. And this part is food, we're Polish at the end of the day. \textit{Heart speaks to heart} but stomachs are louder...
\\
The idea was to include a little bit of everything - Polish cuisine; smart ideas for well-composed lunches that deliver nutrients without burden on you diary; Magda's classic vegan recipes (note on inspiration later); everyday food but also festive\&special meals. In Poland it's common to say that something is 'good as bread' and that home full of love is redolent of bread so we included recipe for sweet buns.\\
We offer you not only this cookbook but also lifelong helpline - if anything is unclear or you'd like more recipes of certain type, you know where to find us.\\
We wish you God's blessing - may He guide you every day and may your love flourish and bear fruit.


\section*{Technology}
We have decided to typeset the book in \LaTeX{} (pronounced as \say{\underline{La} Manche} and \say{\underline{tech}nology}),
a sort of computer language widely used to write scientific publications.
This means that the pdf used to print your hard copy (which is the only one!)
has been generated from multiple files with code specifying the content and layout
of the book.
If you'd ever like to see the source (and previous versions),
you can visit 
% TODO: paste repo link

\LaTeX{}~is slightly different that standard text editors as you don't directly affect the visuals
on the page but rather describe in code what you'd like and the compiler tries to
typeset the page accordingly.
This meanst it is \sout{impossible} hard to cheat and
move the margin a little bit on page 24 because you can't fit something.
We believe that in the long run the coherence of such approach provides aesthetics.

\section*{Licensing}
If the source is not specified, the recipe has been invented by Magda or comes from
ancient traditions of our families.
You may assume we trimmed unreasonable amounts of saturated fats and sugar in these cases.
The outsorced recipes come from different places, most notably www.jadlonomia.com which
provides vegan recipes pionier in Poland and is accompanied by two books by Marta Dymek.
These recipes have been freely interpreted, altered, translated and commented by us. And numerous spices have been added.

All the pictures are our own (sorry for poor quality).
The cover has been designed in canva.com.
You are entitled to arbitrary modification and redistribution of the content and the book itself.
On our side we promise you the uniqueness of the book as a whole of contents and design,
i.e. we won't erase your names from the front page and reuse as a gift for another wedding ;)
However, we keep the right to recycle individual recipes and code,
as well as sharing these with our friends.

Should you need more dead tree copies, please reach out and we are happy to assist with printing.

\section*{Note on strange ingredients}

Some of the ingredients used in this cookbook may not be easily accessible in a typical British grocers'. However, please trust us and admit that usually they are necessary ;). Of course, you can skip some of the spices or ingredients but it won't be the same. Here follows a short justification and advice on where to buy stuff: 

\subsection*{Dried yeast flakes}

Source of umami flavour (fifth basic taste aka savoury taste). Added instead of cheese will enrich taste of the pesto without shortening its life time and increasing fat content (but in calories yes). Rich in group B vitamins, minerals and generally have lots of benefits on your health. 

You can use it for:

\begin{itemize}
    \setlength\itemsep{0.1mm}
    \item pesto;
    \item sandwich spread (eg. blend sunflower seeds, dried tomatoes and olive oil);
    \item soups;
    \item sauces; 
    \item pâté;
    \item just like Parmigiano-Reggiano.
\end{itemize}

(!) It's not the same as dried bakery yeast.
Yeast flakes are deactivated and taste differently.
\\
Probably the easiest way is to buy it on Amazon or shops like Holland \& Barrett.

\subsection* {Smoked paprika}
Similar to aroma of a smoked sausage.
Indispensable for vegetables stews and roast vegetables, particularly aubergine and courgette.
It will also beef omelette up. 
Essential for rich tomato pasta sauces.
Available in Tesco. 

\subsection*{Paprika}
You encounter a differentiation in this cookbook: sweet and hot paprika.
The later one is not to be mistaken with chilli.
Chilli can be super hot, whereas hot paprika is just hot.
For a big pot of tomato sauce you can safely use a teaspoon or even two.
Sweet paprika is a Hungarian spice which can be buy in Tesco under the simple name: paprika.
We call it sweet because it's not spicy.
Nevertheless, as smoked paprika it enriches stews and sauces.
It's far more versatile, though, as it doesn't have very strong flavour.

\section*{Polish food}

\subsection*{Curd cheese aka 'twaróg'}

Not to be confused with quark which is being sold in pots, just like yoghurt.\\
Curd cheese is more dense although you can grind it to get more flowy substance (traditionally, we use classic mincer for this purpose but hand blender can do too).\\
'Twaróg' can be eaten both sweet and savoury. Sweet - in cheesecakes, buns (check the recipe!), crepes etc. Savoury - spinach crepes and sandwich spreads - grind smoked mackerel, egg and curd cheese or curd cheese, sour cream and chives.\\
Available in Polish shops.

\subsection*{Grains}
So, you didn't come to the Polish shop just to buy curd cheese and other dairy products (remember buttermilk!). Grains - this is what you really need. You can buy them in big supermarkets too, but usually the variety is limited and prices are excessive.\\
Grains are staple food. The fact, that most of the people in the UK limit themselves to rice, was a serious amusement. Fry with vegetables, serve with stews, cook 'risotto', prepare stuffed vegetables (eg. roast peppers with barley or buckwheat filling) or simply add to your tasty broth (actually, broth with pearl barley used to be a traditional English soup, correct me if I'm wrong). More schmancy: breakfast, lasagne (see pumpkin millet lasagne) and desserts! More on that in the second edition ;).\\
\textbf{Little glossary:}

\begin{itemize}
    \setlength\itemsep{0.1mm}
    \item buckwheat (kasza gryczana); goes well with spinach and Stilton or roast peppers and carrots. 
    \item millet (kasza jaglana); very subtle, loves being accompanied by cheese and sour cream (+broccoli and carrots or savoy cabbage). Perfect for.... milkshakes (trust me! Cook some millet, add honey, yoghurt and seasonal fruit, blend and breakfast ready!) 
    \item pearl barley; perfect for 'risotto' and broths
    \item barley groats; ideal to eat with stew and meaty sauces. 
\end{itemize}

\section*{Units and abbreviations}
To keep things concise we have used some abbreviations in the ingredient lists:
\begin{table}[h]
\centering
\begin{tabular}{cc}
    c.	& cup                     \\
    tbs. & tablespoon                     \\
    ts.	& teaspoon                    \\
    hfl. & handful (Hugh's) \\
    \multicolumn{1}{l}{} & \multicolumn{1}{l}{}
\end{tabular}
\end{table}