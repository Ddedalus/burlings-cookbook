% Complete recipe example
\begin{recipe}
[% 
    preparationtime = {\unit[30]{min}},
    portion = {\portion{4}},
    source = {Fridge leftovers improvisation}
]
{Green lentil wraps}
    
    \ingredients{%
        8 & tortillas (wholemeal) \\
        \unit[100]{g} & green lentils \\
        1 & green pepper \\
        1 & onion \\
        & green beans \\
        & salad mix \\
        & green peas or mangout \\
        & grated cheese \\
        & salsa sauce
    }
    
    \preparation{%
        \step Boil the lentils until soft with roughly three times as much water (add some salt). Boil the green peas in another pan.
        \step Fry the onion and pepper on medium fire.
        \step Prepare a large, dry and clean, non-stick frying pan and keep it on minimum heat. On a flat surface mix all the ingredient on a tortilla and roll. To keep your fingers clean put the salsa at the bottom and the salad at the top.
        \step Fry the wraps, turning by 90 degrees when they become brown and stiff.
    }
    
    \suggestion[Rolling a wrap]
    {%
        Put the filling only on the bottom half, leaving \unit[2-3]{cm} margin on the sides. With both hands fold around 1/6 of perimeter on each side,
        % TODO: 1/6
        leaving another 1/6 at the bottom. Now holding the sides and pushing them to the middle as you go, roll the bottom over the top. Don't be afraid to compress the ingredients.
    }
    
    \hint{%
        While frying start with the place where the layers meet. Then turn as if you continued to roll it rather than unrolling on the pan!
    }
\end{recipe}